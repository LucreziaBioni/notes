% **************************************************************
% Hi! Edit this file for your presentation!
% **************************************************************

% ==================///==================///==================///
% ==================/// LATEX'S STUFF
% ==================///==================///==================///

\documentclass{beamer}
\usepackage{amsfonts,amsmath,oldgerm}
\usetheme{_statale}
\usefonttheme[onlymath]{serif}
\usepackage{caption} 
\usepackage{xcolor}
\captionsetup[figure]{font=tiny}
\usepackage{tikz}
\usepackage[compat=1.1.0]{tikz-feynman}
\usetikzlibrary{arrows.meta}
\newcommand{\um}{\mathfrak{m}}
\usepackage{stmaryrd}
\usepackage{amsmath,amssymb}
\usepackage{bbold}

\usepackage{ragged2e}

\newcommand{\capcol}[1]{{\color{maincolor}#1}}

\newcommand{\testcolor}[1]{\colorbox{#1}{\textcolor{#1}{test}}~\texttt{#1}}
\newcommand{\hrefcol}[2]{\textcolor{cyan}{\href{#1}{#2}}}
\titlebackground*{assets/background}

% ==================///==================///==================///
% ==================/// SPLASH PAGE
% ==================///==================///==================///

\title{Precision at next-to-leading order: Refining the NSC Subtraction Scheme with $\theta$-Parameters}
\course{Bachelor's Degree in Physics}
\author{\textbf{Lucrezia Bioni}}
\supervisor{\textbf{Prof.~Raoul Horst Röntsch}}
\IDnumber{13655A}
\date{24 October 2025}

% ==================///==================///==================///
% ==================/// START PRESENTATION
% ==================///==================///==================///

\begin{document}
\maketitle

\footlinecolor{maincolor}

% ==================///==================///==================///
% ==================/// BODY'S PRESENTATION
% ==================///==================///==================///

\begin{frame}{The Standard Model of Particle Physics}
  \framesubtitle{-}

  \begin{columns}

    \begin{column}{0.5\textwidth}

      \begin{figure}
        \centering
        \includegraphics[width = \textwidth]{imgs/standard-model.png}
      \end{figure}

    \end{column}

    \begin{column}{0.5 \textwidth}

      \begin{colorblock}[white]{stataleyellow}{Main BSM evidence}
        \begin{itemize}
          \item dark matter and dark energy
          \item matter-antimatter asymmetry
          \item neutrino masses
        \end{itemize}
      \end{colorblock}

      \vspace{0.5em}


      \capcol{Figure} from Clowe et al. 2006.

      \justifying
      Offset between the observed baryonic mass distribution and the gravitational potential in the Bullet Cluster (1E 0657-56).

    \end{column}

  \end{columns}

\end{frame}

%=======================================================================

\begin{frame} {Collider Physics}
  \framesubtitle{One of the principal methods of research}
  
\end{frame}



% ==================///==================///==================///
% ==================/// END PRESENTATION
% ==================///==================///==================///

\end{document}
