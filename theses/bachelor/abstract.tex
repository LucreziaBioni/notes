\selectlanguage{english}
Perturbative Quantum Chromodynamics (QCD) corrections are essential for achieving high-precision theoretical predictions in particle physics, which makes them fundamental for interpreting data from modern particle accelerators, such as the Large Hadron Collider (LHC) at CERN. As experimental precision increases, theoretical predictions must also attain a level of accuracy that allows for meaningful comparisons with data. This requires extending perturbative calculations beyond the leading order (LO) to include NLO and higher-order corrections in the strong coupling constant $\alpha_s$. These corrections account for crucial physical effects, such as real parton emissions and virtual loop contributions. \\
A significant challenge in calculating these higher-order corrections arises from the presence of infrared (IR) singularities, which occur when partons become soft or collinear. Although the sum of all contributions to a cross-section is finite, due to the cancellation of divergences between real emissions and virtual loops, each individual term is divergent. To manage these divergences and obtain finite, numerically stable results, it is necessary to introduce subtraction schemes that regularize the singularities locally in the phase space. \\
Over the past few decades, a variety of subtraction schemes have been developed for this purpose. Among them, the Nested Soft-Collinear (NSC) Subtraction Scheme has gained particular prominence due to its conceptual clarity, modular structure, and consistent extensibility from NLO to next-to-next-to-leading order (NNLO). The NSC scheme achieves a local cancellation of infrared singularities by factorizing soft and collinear limits in a nested manner. Furthermore, its modular nature allows the subtraction terms for complex processes to be built from a limited set of basic components, making it well-suited for implementation in automated Monte Carlo frameworks. \\
The aim of this thesis is to extend the NSC Subtraction Scheme by introducing a set of continuous parameters, called $\theta$-parameters, which systematically restrict the subtraction procedure to the singular, unresolved regions of phase space where soft and collinear singularities occur. Specifically, the parameter $\theta_s$ limits the energy of unresolved partons in the soft limit, while $\theta_i$ controls their angular separation in the collinear limit. The central motivation for this development is to improve numerical stability and efficiency in Monte Carlo integrations, particularly for multi-leg processes where redundant phase-space coverage can significantly slow down convergence. \\
The implementation and analytical study in this thesis demonstrate that introducing $\theta$-parameters preserves the theoretical consistency and modular nature of the NSC scheme. These parameters modify only the soft and collinear (real) contributions, while leaving the virtual corrections and those from PDF renormalization unchanged. Specifically, they appear in the $I_{\mathrm{S}}(\epsilon)$ and $I_{\mathrm{C}}(\epsilon)$ operators, which are part of the soft and collinear counterterms for the infrared-divergent scattering amplitudes. Crucially, the $\theta$-parameters do not compromise the cancellation of the infrared $1/\epsilon^2$ and $1/\epsilon$ poles when all contributions are summed. The coefficient of the $1/\epsilon^2$ pole in $I_{\mathrm{S}}(\epsilon)$ is independent of any parameter, and, while the coefficients of the $1/\epsilon$ poles in $I_{\mathrm{S}}(\epsilon)$ and $I_{\mathrm{C}}(\epsilon)$ depend on $\theta_s$, this dependence cancels out in their sum. This confirms that the modified NSC scheme yields the same physical cross-sections as the standard one. \\
Therefore, the study presented in this thesis contributes to the ongoing effort to improve the efficiency and reliability of higher-order QCD computations. Such improvements are fundamental to continue testing the Standard Model and searching for new Physics beyond its confines.
